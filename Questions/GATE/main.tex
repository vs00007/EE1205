% \iffalse
\let\negmedspace\undefined
\let\negthickspace\undefined
\documentclass[journal,12pt,twocolumn]{IEEEtran}
\usepackage{cite}
\usepackage{amsmath,amssymb,amsfonts,amsthm}
\usepackage{algorithmic}
\usepackage{graphicx}
\usepackage{textcomp}
\usepackage{xcolor}
\usepackage{txfonts}
\usepackage{listings}
\usepackage{enumitem}
\usepackage{mathtools}
\usepackage{gensymb}
\usepackage{comment}
\usepackage[breaklinks=true]{hyperref}
\usepackage{tkz-euclide} 
\usepackage{listings}
\usepackage{gvv}                                        
\def\inputGnumericTable{}                                 
\usepackage[latin1]{inputenc}                                
\usepackage{color}                                            
\usepackage{array}                                            
\usepackage{longtable}                                       
\usepackage{calc}                                             
\usepackage{multirow}                                         
\usepackage{hhline}                                           
\usepackage{ifthen}                                           
\usepackage{lscape}
\newtheorem{theorem}{Theorem}[section]
\newtheorem{problem}{Problem}
\newtheorem{proposition}{Proposition}[section]
\newtheorem{lemma}{Lemma}[section]
\newtheorem{corollary}[theorem]{Corollary}
\newtheorem{example}{Example}[section]
\newtheorem{definition}[problem]{Definition}
\newcommand{\BEQA}{\begin{eqnarray}}
\newcommand{\EEQA}{\end{eqnarray}}
\newcommand{\define}{\stackrel{\triangle}{=}}
\theoremstyle{remark}
\newtheorem{rem}{Remark}
\begin{document}

\bibliographystyle{IEEEtran}
\vspace{3cm}

\title{CE-25}
\author{EE23BTECH11063 - Vemula Siddhartha}
\maketitle
\newpage
\bigskip

\renewcommand{\thefigure}{\theenumi}
\renewcommand{\thetable}{\theenumi}
\textbf{Question}:\\
The following function is defined over the integral $[-L,L]:$
\begin{align}
    f\brak{x}=px^4+qx^5\notag
\end{align}
It is expressed as a Fourier series,
\begin{align}
    f\brak{x}=a_0+\sum_{n=1}^{\infty}\cbrak{a_n\sin\brak{\frac{\pi x}{L}}+b_n\cos\brak{\frac{\pi x}{L}}},\notag
\end{align}
which options amongst the following are true?
\begin{enumerate}
    \item $a_n$, $n=1,2,..,\infty$ depend on $p$
    \item $a_n$, $n=1,2,..,\infty$ depend on $q$
    \item $b_n$, $n=1,2,..,\infty$ depend on $p$
    \item $b_n$, $n=1,2,..,\infty$ depend on $q$
\end{enumerate}
\end{document}