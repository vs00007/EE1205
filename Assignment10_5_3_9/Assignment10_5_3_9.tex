\let\negmedspace\undefined
\let\negthickspace\undefined
\documentclass[journal,12pt,twocolumn]{IEEEtran}
\usepackage{cite}
\usepackage{amsmath,amssymb,amsfonts,amsthm}
\usepackage{algorithmic}
\usepackage{graphicx}
\usepackage{textcomp}
\usepackage{xcolor}
\usepackage{txfonts}
\usepackage{listings}
\usepackage{enumitem}
\usepackage{mathtools}
\usepackage{gensymb}
\usepackage{comment}
\usepackage[breaklinks=true]{hyperref}
\usepackage{tkz-euclide} 
\usepackage{listings}
\usepackage{gvv}                                        
\def\inputGnumericTable{}                                 
\usepackage[latin1]{inputenc}                                
\usepackage{color}                                            
\usepackage{array}                                            
\usepackage{longtable}                                       
\usepackage{calc}                                             
\usepackage{multirow}                                         
\usepackage{hhline}                                           
\usepackage{ifthen}                                           
\usepackage{lscape}

\newtheorem{theorem}{Theorem}[section]
\newtheorem{problem}{Problem}
\newtheorem{proposition}{Proposition}[section]
\newtheorem{lemma}{Lemma}[section]
\newtheorem{corollary}[theorem]{Corollary}
\newtheorem{example}{Example}[section]
\newtheorem{definition}[problem]{Definition}
\newcommand{\BEQA}{\begin{eqnarray}}
\newcommand{\EEQA}{\end{eqnarray}}
\newcommand{\define}{\stackrel{\triangle}{=}}
\theoremstyle{remark}
\newtheorem{rem}{Remark}
\begin{document}

\bibliographystyle{IEEEtran}
\vspace{3cm}

\title{10.5.3.9}
\author{EE23BTECH11063 - Vemula Siddhartha}
\maketitle
\newpage
\bigskip

\renewcommand{\thefigure}{\theenumi}
\renewcommand{\thetable}{\theenumi}
\textbf{Question}:\\
If the sum of first 7 terms of an AP is 49 and that of 17 terms is 289, find the sum of
first n terms.
\\\\
\textbf{Solution: }\\
The sum of first $r$ terms of an Arithmetic Progression (AP) $S_r$ , whose first term is $a$ and common difference is $d$ is:
\begin{align}
S_r=\frac{r}{2}\,(2a+(r-1)\,d)\label{eq1}
\end{align}
Let the given AP have first term $a$ and common difference $d$.\\
Given, the sum of first $7$ terms of the AP is 49.
\begin{align}
S_7&=49\notag\\
49&=\frac{7}{2}\,(2a+(7-1)\,d)\notag\\
49&=\frac{7}{2}\,(2a+6d)\notag\\
a+3d&=7\label{eq2}
\end{align}
Also given, the sum of first $17$ terms of the AP is 289.
\begin{align}
S_{17}&=289\notag\\
289&=\frac{17}{2}\,(2a+(17-1)\,d)\notag\\
289&=\frac{17}{2}\,(2a+16d)\notag\\
a+8d&=17\label{eq3}
\end{align}
Subtracting equation \ref{eq2} from equation \ref{eq3} we get:
\begin{align}
5d&=10\notag\\
d&=2\label{eq4}
\end{align}
Substituting the value of $d$ in equation 1 we get:
\begin{align}
a+6&=7\notag\\
a&=1\label{eq5}
\end{align}
The sum of first n terms of the AP is:
\[S_n= \frac{n}{2}\,(2a+(n-1)\,d)\]
Substituting the values of $a$ and $d$:
\begin{align}
S_n&=\frac{n}{2}\,(2(1)+(n-1)(2))\notag\\
S_n&=n\,(1+n-1)\notag\\
S_n&=n^2\label{eq6}
\end{align}
The signal corresponding to this will be:
\[x(n)= n^2\,u(n)\]
Applying z-transform:
\begin{align}
X(z)&=\sum_{n=-\infty}^{\infty}(n^2\,u(n)\,)z^{-n}\notag\\
X(z)&=0+ \sum_{n=1}^\infty (n^2)\,z^{-n}\notag\\
X(z)&=(1^2)\,z^{-1}+(2^2)\,z^{-2}+(3^2)\,z^{-3}+...\notag\\
X(z)&=z^{-1}+4z^{-2}+9z^{-3}+16z^{-4}+...\label{eq7}
\end{align}
Multiplying the equation \ref{eq7} with $z^{-1}$:\\
\begin{align}
z^{-1}X(z)&=z^{-2}+4z^{-3}+9z^{-4}+16z^{-5}+...\label{eq8}
\end{align}
Subtracting equation \ref{eq8} from equation \ref{eq7}:
\begin{align}
X(z)(1-z^{-1})=z^{-1}+3z^{-2}+5z^{-3}+7z^{-4}+...\label{eq9}
\end{align}
Multiplying the equation \ref{eq9} with $z^{-1}$:
\begin{align}
X(z)(z^{-1}-z^{-2})=z^{-2}+3z^{-3}+5z^{-4}+7z^{-5}+...\label{eq10}
\end{align}
Subtracting equation \ref{eq10} from equation \ref{eq9}:
\begin{align}
X(z)(1-2z^{-1}+z^{-2})&=z^{-1}+2(z^{-2}+z^{-3}+z^{-4}+...)\notag\\
X(z)(1-z^{-1})^2&=z^{-1}+2\frac{(z^{-2})}{(1-z^{-1})}\notag\\
X(z)&=\frac{z^{-1}(1-z^{-1})+2z^{-2}}{(1-z^{-1})^3}\notag\\
X(z)&=\frac{z^{-1}(1+z^{-1})}{(1-z^{-1})^3}
\end{align}

\end{document}
