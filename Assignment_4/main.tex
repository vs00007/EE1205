% \iffalse
\let\negmedspace\undefined
\let\negthickspace\undefined
\documentclass[journal,12pt,twocolumn]{IEEEtran}
\usepackage{cite}
\usepackage{amsmath,amssymb,amsfonts,amsthm}
\usepackage{algorithmic}
\usepackage{graphicx}
\usepackage{textcomp}
\usepackage{xcolor}
\usepackage{txfonts}
\usepackage{listings}
\usepackage{enumitem}
\usepackage{mathtools}
\usepackage{gensymb}
\usepackage{comment}
\usepackage[breaklinks=true]{hyperref}
\usepackage{tkz-euclide} 
\usepackage{listings}
\usepackage{gvv}                                        
\def\inputGnumericTable{}                                 
\usepackage[latin1]{inputenc}                                
\usepackage{color}                                            
\usepackage{array}                                            
\usepackage{longtable}                                       
\usepackage{calc}                                             
\usepackage{multirow}                                         
\usepackage{hhline}                                           
\usepackage{ifthen}                                           
\usepackage{lscape}
\newtheorem{theorem}{Theorem}[section]
\newtheorem{problem}{Problem}
\newtheorem{proposition}{Proposition}[section]
\newtheorem{lemma}{Lemma}[section]
\newtheorem{corollary}[theorem]{Corollary}
\newtheorem{example}{Example}[section]
\newtheorem{definition}[problem]{Definition}
\newcommand{\BEQA}{\begin{eqnarray}}
\newcommand{\EEQA}{\end{eqnarray}}
\newcommand{\define}{\stackrel{\triangle}{=}}
\theoremstyle{remark}
\newtheorem{rem}{Remark}
\begin{document}

\bibliographystyle{IEEEtran}
\vspace{3cm}

\title{IN-37}
\author{EE23BTECH11063 - Vemula Siddhartha}
\maketitle
\newpage
\bigskip

\renewcommand{\thefigure}{\theenumi}
\renewcommand{\thetable}{\theenumi}
\textbf{Question}:\\
The signal flow graph of a system is shown. The expression for $\frac{Y\brak{s}}{X\brak{s}}$ is
\begin{figure}[h]
    \centering
    \begin{tikzpicture}
    \draw (0,0) arc (0:360:0.1);
    \draw [->] (0,0) -- (2,0);
    \draw (2.2,0) arc (0:360:0.1);
    \draw [->] (2.2,0) -- (4.2,0);
    \draw (4.4,0) arc (0:360:0.1);
    \draw [->] (4.4,0) -- (6.4,0);
    \draw (6.6,0) arc (0:360:0.1);
    \draw (-0.5,0) node[below] {$X\brak{s}$};
    \draw (1.1, 0) node[above] {$G_1\brak{s}$};
    \draw (3.3,0) node[above]  {$G_2\brak{s}$};
    \draw (5.3, 0) node[above] {$2$};
    \draw (6.9,0) node[below] {$Y\brak{s}$};
    \draw [<-](4.3,0.1) arc (0:180:1.1);
    \draw [<-] (2.1, -0.1) arc (-180:0:1.1);
    \draw  (3.2,1.3) node[above] {$G_3\brak{s}$};
    \draw (3.2,-1.2) node[below] {-1};
    \draw (2,-0.3) node[left] {$a$};
    \draw (4.3, -0.3) node[right] {$b$};
    \end{tikzpicture}
    \caption{Signal Flow Graph of the System}
    \label{fig:sfg_in-37-2022}
\end{figure}
\begin{enumerate}[label=(\alph*)]
    \item $\dfrac{2 G_1\brak{s} G_2\brak{s} + 2 G_1\brak{s} G_3\brak{s} }{ 1 + G_2\brak{s} + G_3\brak{s} }$
    \item $ 2 + G_1\brak{s} + G_3\brak{s} + \dfrac{G_2\brak{s} }{ 1 + G_2\brak{s}}$
    \item $G_1\brak{s} + G_3\brak{s} - \dfrac{G_2\brak{s} }{ 2 + G_2\brak{s} }$
    \item $\dfrac{ 2 G_1\brak{s} G_2\brak{s} + 2 G_1\brak{s} G_3\brak{s} - G_1\brak{s} }{ 1 + G_2\brak{s} + G_3\brak{s} }$
\end{enumerate}\hfill(GATE 2022 IN Question 37) \\
\solution
\begin{table}[h!]    
    \centering
    \begin{tabular}[12pt]{ |c| c|}
    \hline
    \textbf{Symbol} & \textbf{Description}\\ 
    \hline
    $ f\brak{x} $ & Function \\
    \hline 
    $ G\brak{f} $ & Fourier Transform of the function $ f\brak{x} $ \\
    \hline
    $ f^*\brak{x} $ & Complex Conjugate of $ f\brak{x} $ \\
    \hline
    $ G^*\brak{f} $ & Complex Conjugate of $ G\brak{f} $ \\
    \hline
    Im$ \brak{ G \brak{ f } }$ & Imaginary Part of $ G\brak{f} $ \\
    \hline
    \end{tabular}
    \caption{Variables Used}
  \end{table}\\
  \begin{align}
    P_1 &= \brak{G_1\brak{s}} \brak{G_2\brak{s}} \brak{2} = 2 G_1\brak{s} G_2\brak{s} \\
    P_2 &= \brak{G_1\brak{s}} \brak{G_3\brak{s}} \brak{2} = 2 G_1\brak{s} G_3\brak{s} \\
    \Delta_1 &= 1 - \brak{0} = 1 \\
    \Delta_2 &= 1 - \brak{0} = 1 \\
    L_1 &= -G_2\brak{s} \\
    L_2 &= -G_3\brak{s} \\
    \Delta &= 1 - \brak{L_1 + L_2} = 1 + G_1\brak{s} + G_2\brak{s}
  \end{align}
  From \ref{fig:sfg_in-37-2022}, Using Mason's Gain Formula,
  \begin{align}
    \frac { Y\brak{s} }{ X\brak{s} } &= \frac { \sum_{ i = 1 }^{ n } P_i \Delta_i } { \Delta } \\
    &= \frac { P_1 \Delta_1 + P_2 \Delta_2 } { \Delta } \\
    &= \frac { 2 G_1\brak{s} G_2 \brak{s} \brak{1} + 2 G_1\brak{s} G_3\brak{s} \brak{1} } { 1 + G_2\brak{s} + G_3\brak{s} } \\
    \implies \frac { Y\brak{s} }{ X\brak{s} } &= \frac { 2 G_1\brak{s} G_2\brak{s} + 2 G_1\brak{s} G_3\brak{s} } { 1 + G_2\brak{s} + G_3\brak{s} }
  \end{align}
\end{document}
