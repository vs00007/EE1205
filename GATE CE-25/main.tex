% \iffalse
\let\negmedspace\undefined
\let\negthickspace\undefined
\documentclass[journal,12pt,twocolumn]{IEEEtran}
\usepackage{cite}
\usepackage{amsmath,amssymb,amsfonts,amsthm}
\usepackage{algorithmic}
\usepackage{graphicx}
\usepackage{textcomp}
\usepackage{xcolor}
\usepackage{txfonts}
\usepackage{listings}
\usepackage{enumitem}
\usepackage{mathtools}
\usepackage{gensymb}
\usepackage{comment}
\usepackage[breaklinks=true]{hyperref}
\usepackage{tkz-euclide} 
\usepackage{listings}
\usepackage{gvv}                                        
\def\inputGnumericTable{}                                 
\usepackage[latin1]{inputenc}                                
\usepackage{color}                                            
\usepackage{array}                                            
\usepackage{longtable}                                       
\usepackage{calc}                                             
\usepackage{multirow}                                         
\usepackage{hhline}                                           
\usepackage{ifthen}                                           
\usepackage{lscape}
\newtheorem{theorem}{Theorem}[section]
\newtheorem{problem}{Problem}
\newtheorem{proposition}{Proposition}[section]
\newtheorem{lemma}{Lemma}[section]
\newtheorem{corollary}[theorem]{Corollary}
\newtheorem{example}{Example}[section]
\newtheorem{definition}[problem]{Definition}
\newcommand{\BEQA}{\begin{eqnarray}}
\newcommand{\EEQA}{\end{eqnarray}}
\newcommand{\define}{\stackrel{\triangle}{=}}
\theoremstyle{remark}
\newtheorem{rem}{Remark}
\begin{document}

\bibliographystyle{IEEEtran}
\vspace{3cm}

\title{CE-25}
\author{EE23BTECH11063 - Vemula Siddhartha}
\maketitle
\newpage
\bigskip

\renewcommand{\thefigure}{\theenumi}
\renewcommand{\thetable}{\theenumi}
\textbf{Question}:\\
The following function is defined over the interval $[-L,L]$:
    $$f\brak{x}=px^4+qx^5$$
It is expressed as a Fourier series,
    $$f\brak{x}=a_0+\sum_{n=1}^{\infty}\cbrak{a_n\sin\brak{\frac{\pi n x}{L}}+b_n\cos\brak{\frac{\pi n x}{L}}}$$
which options amongst the following are true?
\begin{enumerate}[label=(\alph*)]
    \item $a_n$, $n=1,2,..,\infty$ depend on $p$
    \item $a_n$, $n=1,2,..,\infty$ depend on $q$
    \item $b_n$, $n=1,2,..,\infty$ depend on $p$
    \item $b_n$, $n=1,2,..,\infty$ depend on $q$
\end{enumerate}
\textbf{Solution:}
\begin{table}[h!]    
    \centering
    \begin{tabular}[12pt]{ |c| c|}
    \hline
    \textbf{Parameter} & \textbf{Description}\\ 
    \hline
    $f\brak{x}$ & Polynomial function\\
    \hline
    $2L$& Period of the Polynomial function\\ 
    \hline
    $c_n$ & Complex Fourier Coefficients\\
    \hline
    $a_0,\,a_n,\,b_n$& Trigonometric Fourier Coefficients\\
    \hline   
    \end{tabular}
    \caption{Parameters}
    \label{tab:CE:25}
\end{table}\\
The complex exponential Fourier Series of $f\brak{x}$ is,
\begin{align}
    f\brak{x}&=\sum_{n=-\infty}^{\infty}c_ne^{j\frac{\pi nx}{L}}\\
    \implies c_n&=\frac{1}{2L}\int_{-L}^{L}f\brak{x}e^{-j\frac{\pi n x}{L}}\;dx\\
    c_n&=\frac{1}{2L}\int_{-L}^{L}\brak{px^4+qx^5}e^{-j\frac{\pi n x}{L}}\;dx
\end{align}
For $n=0$, 
\begin{align}
    c_0&=\frac{1}{2L}\int_{-L}^{L}\brak{px^4+qx^5}\;dx\\
    &=\frac{pL^4}{5}
\end{align}
For $n\neq0$,
\begin{align}
    c_n&=\frac{1}{2L}\int_{-L}^{L}\brak{px^4+qx^5}e^{-j\frac{\pi n x}{L}}\;dx\label{eq:CE-25.1}\\
    &=\frac{pL^4}{2}\brak{e^{j\pi n}-e^{-j\pi n}}\brak{\frac{1}{j\pi n}+\frac{12}{\brak{j\pi n}^3}+\frac{24}{\brak{j\pi n}^5}}\notag\\
    &\;\;\;-\frac{pL^4}{2}\brak{e^{j\pi n}+e^{-j\pi n}}\brak{\frac{4}{\brak{j\pi n}^2}+\frac{24}{\brak{j\pi n}^4}}\notag\\
    &\;\;\;-\frac{qL^5}{2}\brak{e^{j\pi n}+e^{-j\pi n}}\brak{\frac{1}{j\pi n}+\frac{20}{\brak{j\pi n}^3}+\frac{120}{\brak{j\pi n}^5}}\notag\\
    &\;\;\;+\frac{qL^5}{2}\brak{e^{j\pi n}-e^{-j\pi n}}\brak{\frac{5}{\brak{j\pi n}^2}+\frac{60}{\brak{j\pi n}^4}+\frac{120}{\brak{j\pi n}^6}}\\
    &=\brak{pL^4}\brak{-1}^n\brak{\frac{4}{\brak{\pi n}^2}-\frac{24}{\brak{\pi n}^4}}\notag\\
    &\;\;\;-\brak{qL^5}\brak{-1}^n\brak{-\frac{j}{{\pi n}}+\frac{20j}{\brak{\pi n}^3}-\frac{120j}{\brak{\pi n}^5}}
\end{align}
Given,
\begin{align}
    f\brak{x}=a_0+\sum_{n=1}^{\infty}\cbrak{a_n\sin\brak{\frac{\pi n x}{L}}+b_n\cos\brak{\frac{\pi n x}{L}}}
\end{align}
Finding the Fourier Coefficient $a_0$,
\begin{align}
    a_0&=c_0\\
    \implies a_0&=\frac{pL^4}{5}
\end{align}
We know,
\begin{align}
    \sin \theta&=\frac{e^{j\theta}-e^{-j\theta}}{2j}
\end{align}
Finding the Fourier Coefficients $a_n$,
\begin{align}
    a_n&=\frac{1}{L}\int_{-L}^{L}f\brak{x}\sin\brak{\frac{\pi nx}{L}}\;dx\\
    a_n&=\frac{1}{L}\int_{-L}^{L}f\brak{x}\brak{\frac{e^{j\frac{\pi n x}{L}}-e^{-j\frac{\pi n x}{L}}}{2j}}\;dx\\
    &=\frac{1}{2Lj}\int_{-L}^{L}f\brak{x}e^{j\frac{\pi n x}{L}}\;dx-\frac{1}{2Lj}\int_{-L}^{L}f\brak{x}e^{-j\frac{\pi n x}{L}}\;dx
\end{align}
\begin{align}
    \implies a_n&=\frac{c_{-n}-c_n}{j}\\
    a_n&=\brak{-2qL^5}\brak{-1}^{n}\brak{\frac{1}{\pi n}-\frac{2}{\brak{\pi n}^3}+\frac{120}{\brak{\pi n}^5}}
\end{align}
We know,
\begin{align}
    \cos\theta&=\frac{e^{j\theta}+e^{-j\theta}}{2}
\end{align}
Finding the Fourier Coefficients $b_n$,
\begin{align}
    b_n&=\frac{1}{L}\int_{-L}^{L}f\brak{x}\cos\brak{\frac{\pi nx}{L}}\;dx\\
    b_n&=\frac{1}{L}\int_{-L}^{L}f\brak{x}\brak{\frac{e^{j\frac{\pi n x}{L}}+e^{-j\frac{\pi n x}{L}}}{2}}\;dx\\
    &=\frac{1}{2L}\int_{-L}^{L}f\brak{x}e^{j\frac{\pi n x}{L}}\;dx+\frac{1}{2L}\int_{-L}^{L}f\brak{x}e^{-j\frac{\pi n x}{L}}\;dx\\
    \implies b_n&=c_{-n}+c_n\\
    b_n&=\brak{2pL^4}\brak{-1}^n\brak{\frac{4}{\brak{\pi n}^2}-\frac{24}{\brak{\pi n}^4}}
\end{align}
Hence, options \brak{\text{b}} and \brak{\text{c}} are correct.
\begin{figure}[h!]
    \centering
    \includegraphics[width=\linewidth]{figs/Figure_1.png}
    \caption{Fourier Series Approximation of $f\brak{x}$ for $p=5$, $q=2$, $L=0.08$}
\end{figure}
\end{document}