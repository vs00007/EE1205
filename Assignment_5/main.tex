% \iffalse
\let\negmedspace\undefined
\let\negthickspace\undefined
\documentclass[journal,12pt,twocolumn]{IEEEtran}
\usepackage{cite}
\usepackage{amsmath,amssymb,amsfonts,amsthm}
\usepackage{algorithmic}
\usepackage{graphicx}
\usepackage{textcomp}
\usepackage{xcolor}
\usepackage{txfonts}
\usepackage{listings}
\usepackage{enumitem}
\usepackage{mathtools}
\usepackage{gensymb}
\usepackage{comment}
\usepackage[breaklinks=true]{hyperref}
\usepackage{tkz-euclide} 
\usepackage{listings}
\usepackage{gvv}                                        
\def\inputGnumericTable{}                                 
\usepackage[latin1]{inputenc}                                
\usepackage{color}                                            
\usepackage{array}                                            
\usepackage{longtable}                                       
\usepackage{calc}                                             
\usepackage{multirow}                                         
\usepackage{hhline}                                           
\usepackage{ifthen}                                           
\usepackage{lscape}
\newtheorem{theorem}{Theorem}[section]
\newtheorem{problem}{Problem}
\newtheorem{proposition}{Proposition}[section]
\newtheorem{lemma}{Lemma}[section]
\newtheorem{corollary}[theorem]{Corollary}
\newtheorem{example}{Example}[section]
\newtheorem{definition}[problem]{Definition}
\newcommand{\BEQA}{\begin{eqnarray}}
\newcommand{\EEQA}{\end{eqnarray}}
\newcommand{\define}{\stackrel{\triangle}{=}}
\theoremstyle{remark}
\newtheorem{rem}{Remark}
\begin{document}

\bibliographystyle{IEEEtran}
\vspace{3cm}

\title{PH-26}
\author{EE23BTECH11063 - Vemula Siddhartha}
\maketitle
\newpage
\bigskip

\renewcommand{\thefigure}{\theenumi}
\renewcommand{\thetable}{\theenumi}
\textbf{Question}:\\
If $G\brak{f}$ is the Fourier Transform of $f\brak{x}$, then which of the following are true?
\begin{enumerate}[label=(\alph*)]
    \item $G\brak{-f}=+G^*\brak{f}$ implies $f\brak{x}$ is real.
    \item $G\brak{-f}=-G^*\brak{f}$ implies $f\brak{x}$ is purely imaginary.
    \item $G\brak{-f}=+G^*\brak{f}$ implies $f\brak{x}$ is purely imaginary.
    \item $G\brak{-f}=-G^*\brak{f}$ implies $f\brak{x}$ is real.
\end{enumerate}
\hfill(GATE 2022 PH Question 26)\\
\solution
\begin{table}[h!]    
    \centering
    \begin{tabular}[12pt]{ |c| c|}
    \hline
    \textbf{Parameter} & \textbf{Description}\\ 
    \hline
    $f\brak{x}$ & Polynomial function\\
    \hline
    $2L$& Period of the Polynomial function\\ 
    \hline
    $c_n$ & Complex Fourier Coefficients\\
    \hline
    $a_0,\,a_n,\,b_n$& Trigonometric Fourier Coefficients\\
    \hline   
    \end{tabular}
    \caption{Given Information}
  \end{table}
\begin{align}
    f\brak{x} &\system{F} G\brak{f} \\
    G\brak{f} &= \int_{-\infty}^{\infty} f\brak{x} e^{-j 2 \pi f x}\;dx \\
    \implies G\brak{-f} &= \int_{-\infty}^{\infty} f\brak{x} e^{j 2 \pi f x}\; dx \label{eq:PH-26-2022.1} \\
    \implies G^*\brak{f} &= \int_{-\infty}^{\infty} f^*\brak{x} e^{j 2 \pi f x}\; dx \label{eq:PH-26-2022.2}
\end{align}
If $G\brak{-f} = + G^*\brak{f}$, from \eqref{eq:PH-26-2022.1} and \eqref{eq:PH-26-2022.2},
\begin{align}
    f\brak{x} = f^*\brak{x}
\end{align}
Hence, $f\brak{x}$ is real. \\
If $G\brak{-f} = - G^*\brak{f}$, from \eqref{eq:PH-26-2022.1} and \eqref{eq:PH-26-2022.2},
\begin{align}
    f\brak{x} = -f^*\brak{x}
\end{align}
Hence, $f\brak{x}$ is purely imaginary. \\
Therefore, $\brak{\text{a}}$ and $\brak{\text{b}}$ are true.
\end{document}
